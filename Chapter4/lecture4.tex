\documentclass{beamer}
\usepackage{listings}
\begin{document}
\title{Javascript Chapter 4: Variables, Arrays, and Javascript in React}
\author{Ezequiel Torres}
\date{\today}
\frame{\titlepage}
\frame{\frametitle{Table of contents}\tableofcontents}

\section{Basics of Javascript}
\frame{\frametitle{Basics of Javascript}
    Now we have a button and a style, but no way to format it. When wanting to move the elements around the page, we can use a
    style sheet!
}

\begin{frame}[fragile]
\subsection{My Button Component Home}
\frametitle{My Button Component Home}
\begin{lstlisting}
export default function Home() {
  return (
  <>
    <img src=
    "https://image.petmd.com/files/styles/863x625/public/CANS_dogsmiling_379727605.jpg"/>
  </>
 );
}
\end{lstlisting}
\end{frame}

\section{Basics of Classnames}
\frame{\frametitle{Basics of Classnames}
    We can add a classname to our HTML tag to add a stylesheet to it    
}

\begin{frame}[fragile]
\subsection{My Button Stylesheet}
\frametitle{My Button Stylesheet}
\begin{lstlisting}
/* In your CSS */
.avatar {
  border-radius: 50%;
}

export default function Home() {
  return (
  <>
    <img className="avatar" src="https://image.petmd.com/files/styles/863x625/public/CANS_dogsmiling_379727605.jpg"/>
  </>
 );
}
\end{lstlisting}
\end{frame}

\section{CSS Properties}
\frame{\frametitle{CSS Properties}
    \href{https://www.w3schools.com/cssref/css4_pr_accent-color.php}CSSWebpage
}
\end{document}
