\documentclass{beamer}
\usepackage{listings}
\begin{document}
\title{Javascript Chapter 4: Variables, Arrays, and Javascript in React}
\author{Ezequiel Torres}
\date{\today}
\frame{\titlepage}
\frame{\frametitle{Table of contents}\tableofcontents}

\section{Variables}
\frame{\frametitle{Basics of Javascript}
    Now that we have styles and html, we want to start writing code for the front end
}


\frame{\frametitle{Basics of Javascript cont.}
    Let's start with the basics, variables
}

\subsection{Variables Example}
\begin{frame}[fragile]
\subsection{Javascript Calculator Example}
\frametitle{Javascript Calculator Example}
\begin{lstlisting}
function sumTwoNumbers(a: number, b: number): number {
    return a + b
}
\end{lstlisting}
\end{frame}

\begin{frame}[fragile]
\subsection{Javascript Calculator Example cont.}
\frametitle{Javascript Calculator Example cont.}
\begin{lstlisting}
export default function Home(){
    return (
        <div className={styles.main}>
            {sumTwoNumbers(10, 5)}
        </div>
    );
}
\end{lstlisting}
\end{frame}

\subsection{Variables and Typescript}
\frame{\frametitle{Variables and Typescript}
    Now we have a webpage with a simple 15. Notice how we have parameters, inputs for our functions, with a colon 
    and it's type (a number). This is typescript, letting our future selfs know the types of these variables
}

\section{Arrays}
\frame{\frametitle{Arrays}
    When we want to have a collection of data, we can utilize javascript arrays. Let's change our example to 
    sum all the numbers in an array
}

\subsection{Arrays Example}
\begin{frame}[fragile]
\subsection{Arrays Example}
\frametitle{Arrays Example}
\begin{lstlisting}
function sumNumbers(a: number[]) {
    let sum: number = 0
    for (let item of a){
        sum += item
    }
    return sum
}
\end{lstlisting}
\end{frame}

\begin{frame}[fragile]
\frametitle{Arrays Example cont.}
\begin{lstlisting}
export default function Home(){
    return (
        <div className={styles.main}>
            {sumNumbers([5, 10, 15])}
        </div>
    );
}
\end{lstlisting}
\end{frame}

\subsection{Other sum array examples}
\frame{\frametitle{Other sum array examples}
    There are multiple ways to solve the above problem.
}

\begin{frame}[fragile]
\frametitle{Sum using reduce}
\begin{lstlisting}
export default function Home(){
    return (
        <div className={styles.main}>
            {[5, 10, 15].reduce((sum, element) => {return sum + element}, 0)}
        </div>
    );
}
\end{lstlisting}
\end{frame}


\end{document}
