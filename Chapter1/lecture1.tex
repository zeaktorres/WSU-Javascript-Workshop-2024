\documentclass{beamer}
\usepackage{listings}
\begin{document}
\title{Javascript Chapter 1}
\author{Ezequiel Torres}
\date{\today}
\frame{\titlepage}
\frame{\frametitle{Table of contents}\tableofcontents}

\section{Before getting started}
\frame{\frametitle{Before getting started}
    We will be going over the basics of Javascript and developiong in React in this course.
    Here are some great resources to get started.
    \begin{itemize}
        \item<1-> \href{https://react.dev/learn}React.dev Learn
        \item<2-> \href{https://nextjs.org/learn/}Next JS Learn
        \item<3-> \href{https://www.w3schools.com/nodejs/nodejs_mysql.asp}Node JS SQL
    \end{itemize}
}

\section{Environment Setup}
\subsection{Environment Setup NVM}
\frame{\frametitle{Environment setup NVM}
    Here are the basic steps to running python. 
    \vspace{\baselineskip}

    On a Windows PC, you will need NVM to download React: \href{https://github.com/coreybutler/nvm-windows}NVM Windows
    \vspace{\baselineskip}

    On Linux, you will run the following command `curl -o- https://raw.githubusercontent.com/nvm-sh/nvm/v0.39.7/install.sh | bash`  
    Then, `source ~/.bashrc` if shell is bash (use `which \$SHELL`) to see which shell you have 
    \vspace{\baselineskip}
}

\subsection{Environment Setup IDE}
\frame{\frametitle{IDE}
    \begin{enumerate}
        \item<1-> VSCode, a very popular text editor in electron. Great for industry practice
        \item<1-> Neovim/Vim, I personally use Neovim because I hate myself
    \end{enumerate}
}

\subsection{Environment Setup NPM}
\frame{\frametitle{Environment Setup NVM}
    Once you have NVM, you can run `nvm install 20` in the terminal to install node, which will allow you to install Next JS
}

\subsection{Environment Setup Next JS}
\frame{\frametitle{Environment Setup Next JS}
    Now that we have node, it runs Javascript in the terminal, now we can grab a React framework. We will go with Next JS, which has a 
    javascript backend and frontend. 
    \vspace{\baselineskip}

    Run the command `npx create-next-app@latest` and use the next slides options
}

\subsection{Environment Setup Next JS Options} 
\frame{\frametitle{Environment Setup Next Options}
     What is your project named? … wsu-example-project
     Would you like to use TypeScript? … No / Yes
     Would you like to use ESLint? … No / Yes
     Would you like to use Tailwind CSS? … No / Yes
     Would you like to use `src/` directory? … No / Yes
     Would you like to use App Router? (recommended) … No / Yes
     Would you like to customize the default import alias (@/*)? … No / Yes
    Creating a new Next.js app in /home/zeak/WSU-Javascript-Workshop-2024/wsu-example-project.
}

\subsection{Environment Setup Next JS Options} 
\frame{\frametitle{Environment Setup Next Options}
    Now we run the server which hosts the webpage at localhost:3000. `npm run dev`
}

\section{Live Reloading}
\subsection{Live Reloading}
\frame{\frametitle{Live Reloading}
    Live Reloading occurs when you update the page. You should be able to see your changes live. Try changing the page tsx to the below example code
}

\section{Page TSX Basic Example}
\subsection{page.tsx}
\frame{\frametitle{page.tsx}
    Remove all of src/app/page.tsx until it is a blank component like the following slide
}

\begin{frame}[fragile]
\subsection{Page TSX Basic Example Code}
\frametitle{Page TSX Basic Example Code}
\begin{lstlisting}
import Image from "next/image";
import styles from "./page.module.css";

export default function Home() {
  return (
    <main className={styles.main}>
        Test
    </main>
  );
}
\end{lstlisting}
\end{frame}

\end{document}
